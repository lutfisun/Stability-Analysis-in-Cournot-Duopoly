\documentclass{article}
\usepackage[utf8]{inputenc}
\usepackage[english]{babel}
\usepackage{amsthm}

\title{MATH 4394 | Initial Report \\ Stability Analysis in Cournot Duopoly }
\author{Student: Lutfi Sun \\ 
        Advisor: Dr. Saber Elaydi \\ 
        Category: Interdisciplinary Project}
\date{Jan 31, 2020}

\begin{document}

\maketitle

\section{Brief Description}

    Cournot Duopoly is a competition game with two firms whose strategies are the quantities they produce of identical products. This is the benchmark for cournot competition; there are versions with more than two firms and differentiable products. In this project, I will work on finding nash equilibria in cournot duopoly games with multiple stages under Dr. Elaydi's supervision.
    \\ \\
    Nash equilibrium in this context means an outcome where neither of the firms benefit from changing the quantity they produce. I will look at if and how the initial conditions and production functions impact the equilibrium (ie fixed point) firms end up in and what that means in applied economics. I will analyze the existence, uniqueness, and stability of nash equilibria.
    \\ \\
    I will be referring to and using game theory, difference equations, linear algebra, optimization, and real analysis. I am planning to produce phase space diagrams to show the trajectory of the market given initial conditions and find relevant market data.

\section{Motivation and Goals}

    I am an economics double major and chose this project following Dr. Elaydi's guidance. I will get to learn more about analysis and difference equations and apply my mathematics knowledge to economics, which is the field I want to pursue a graduate degree in after Trinity. I am already learning a lot from Dr. Elaydi and the papers I am reading. I believe it will be a challenging and fruitful experience.

\section{Proposed Evaluation}

    I propose the project to be evaluated based on three categories: understanding and use of mathematical concepts, coding in R to visualize the existence and stability of fixed points, and the interpretation of results. I am planning to work with both Dr. Elaydi and Economics faculty to make the project consistent and sound in its mathematics and relevant and applicable to economics.
    
\section{Papers of Inspiration}

    \begin{enumerate}
        \item Bischi, Gian Italo and Michael Kopel (2001). "Equilibrium selection in a nonlinear duopoly game with adaptive expectations ". The Journal of Economic Behavior & Organization. Vol. 46 (2001) 73–100.
    \end{enumerate}
    

\end{document}
